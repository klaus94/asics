\section{Future Work}

Our research has shown: Testing self-adaptive systems is very difficult but good methods were needed for present-day systems.
Currently there are two main-approaches: self-testing and modelbased testing. Each of them has its own advantages and disadvantages. Self-testing is good at handling adaptions and control the system at runtime. On the other hand, it is difficult to ensure good behavior before deploy. Modelbased testing is good at asserting good system behavior prior to release, but can not control the system at runtime.

A combination of both methods could lead to a new testing-method that would combine the advantages of both approaches. Moreover the approaches could benefit from each other.
The model of the system that is created for the modelbased testing approach could also be used as an input for the testmanager, if it contained runtime constraints. This would lower the testing-costs for the testmanager.
Additionaly the monitored data of the self-testing method could be stored and reused as test-data for modelbased tests.