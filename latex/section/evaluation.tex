\section{Evaluation}
\subsection{Criteria}
For our evaluation of the different testing strategies, we need some criteria. With the help of these criteria we can rate and compare the strategies. \\
Below is the list of our criteria with explanations:
\begin{itemize}
\item control during runtime... amount of impact of the test-suite during execution
\item ensure quality before deploy... how much quality can the test-suite encure before the system gets deployed
\item performance overhead... amount of additional effort for running the test-suite simultanously to the system, which includes time and memory
\item testing-cost... how complex is building the test for the developer
\item adaptability... how easy can the test-suite be adapted to an other system
\end{itemize}

\subsection{Comparison}

\subsubsection{Testmanager}
\begin{figure}[h]
	\includegraphics[height=7cm]
	{images/testmanager.png}
	\caption{testmanager - Kiviat graph}
	\label{testmngrKiviat}
\end{figure}
At first we want to evaluate the testmanager approach.
A testmanager has the ability to handle adaptation of the system and therefore its runtime control is high. It has not the best possible score, because there are situation, where the manager can not handle a test violoation (e.g. injured response time constraint). On the other hand it can not ensure good behavior prior to release, because there are no tests before deployment. All tests were performed at runtime, which causes an overhead. By using the ``safe adaption with validation`` method there is a higher time overhead and by using the ``replication with validation`` methode there is a bigger memory overhead. Due to its independance from the monitored system, the testmanager is relatively flexible and can be used as an independent component. A system developer would need to define the constraints for each situation and the testmanager would automatically perform the required tests at runtime. Defining these constraints can be very time-consuming, if there are multiple different environments and actors that need to be included in the testing process.


\subsubsection{Corridor enforcing infrastructure}
\begin{figure}[h]
	\includegraphics[height=7cm]
	{images/corridor.png}
	\caption{corridor enforcing infrastructure - Kiviat graph}
	\label{corridorKiviat}
\end{figure}

\noindent+ An advantage of this method is, that testing the corridor enforcing infrastructure would be enough to guarantee good system behavior. The system itself does not need to be tested.\\
+ testing the CEI is easier than testing the system itself\\
- big infrastructure\\
- overhead at runtime


\subsubsection{modelbased testing}
\begin{figure}[h]
	\includegraphics[height=7cm]
	{images/modelBased.png}
	\caption{model based testing - Kiviat graph}
	\label{modelbasedKiviat}
\end{figure}

