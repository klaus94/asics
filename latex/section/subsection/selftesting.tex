\subsection{Self-Testing}

basic idea...
- monitor system-components and if constraint violated -> do recovery-action
- constraint can be a result or a QoS-constraint (e.g. response time)
- keeping the monitored data can be a good source for later offline-tests

 
\subsubsection{Testmanager}
One idea for a self-test architecture is a testmanager. In general this is a software-component, that runs simultanously to the monitored system. \\
There are two forms of using a testmanager:\\\\

1. safe adaption with validation\\
Every time the system perceives a contextual change it notifys(?) an internal adaptation-manager. It decides wheather an adaption is needed. If the system wants to adapt, the adaptation-manager will initiate an adaption and at the same time notify the testmanager.
After the adaption completed, all actions targeting the system were blocked.
In the meanwhile the testmanager is executing a set of tests that depend on requirements that the adaption-manager sent.
When all tests were finished the result is sent back to the system. The systems adaptation-manager will keep the changes, if the tests were successful or it will recover the old system-state, if a test failed.\\\\

2. replication with validation
The main idea of this architecture is similar to the idea of safe adaption with validation.
When the system needs to do an adaptation it notifies a test-manager. But in contrast


 
 
\subsubsection{Corridor Enforcing Infrastructure}



